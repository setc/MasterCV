%----------------------------------------------------------------------------------%
%   Sebastian Torrente's Master Curriculum Source								   %
%----------------------------------------------------------------------------------%
%
%	Given that I should taylor my CV according to each job offer I decided to
%	create a Master CV source file with absolutely everything I can put and then
%   comment whatever is not needed in each case. 
%
%	Based on the template by Xavier Danaux (xdanaux@gmail.com) and using the 
%	moderncv package. Copyright 2006-2012
%
% 	This work may be distributed and/or modified under the
% 	conditions of the LaTeX Project Public License version 1.3c,
% 	available at http://www.latex-project.org/lppl/.
%
%	1.1 Major Revision: Addendums
%		New idea right here: creating a serie of independet "subcurriculums" called
%		addendums where I can put everything in case future contractors want to 
%		search even further. This save space in the CV sent and, again, allow them
%		take a more detailed look of all the things I've done.
%
%		So far I'm going to make three: one for additional education, other one for
%		experience and a third one with personal skills and other achievement.
%
%   1.2 Rewriting:
%		After the addition of the profile, I commented a lot of things and moved 
%		the	some parts from experience to education. Also changed the tone. 
%		Seriously, doing a cv is way more satisfying and fullfilling when you talk 
%		with your own voice.
%		Just wait when all this thing will be completed, is going to be amazing.
%
%----------------------------------------------------------------------------------%
%   Index																	[Ind]  %
%----------------------------------------------------------------------------------%
%	This is a small index to jump to each part of the document.
%	
%	Index						-- [Ind]
%	Options						-- [Opt]
%	Theme						-- [The]
%	Personal Data				-- [Prd]
%	Content						-- [Con]
%		Profile					-- [Pro]
%		Education				-- [Edu]
%		Additional Education	-- [aEdu]
%		Experience				-- [Exp]
%		Languages				-- [Lang]
%		Computer Skills			-- [Com]
%		Interests				-- [Int]
%   	Personal Skills			-- [Per]
%
%----------------------------------------------------------------------------------%
%   Options 																[Opt]  %
%----------------------------------------------------------------------------------%
%   Fontsize: ('10pt', '11pt' and '12pt')
%   Papersize: ('a4paper', 'letterpaper', 'a5paper', 'legalpaper', 'executivepaper',
%		'landscape')
%   Type of font: ('sans' y 'roman')
\documentclass[11pt,a4paper,roman]{moderncv}

%----------------------------------------------------------------------------------%
%   Theme 																	[The]  %
%----------------------------------------------------------------------------------%

%	Options ('casual' (default),'classic', 'oldstyle', 'banking')
\moderncvstyle{casual} 

%	Colors ('blue' (default), 'orange', 'green', 'red', 'purple', 'grey' y 'black')
\moderncvcolor{black}

%	Fonts:
% \sdefault:  sans serif font
% \rmdefault: roman font
\renewcommand{\familydefault}{\rmdefault}
%\renewcommand{\familydefault}{\sdefault}

%	Number the pages?
\nopagenumbers{}                             

%	Codification:
%\usepackage[utf8]{inputenc}
%\usepackage{CJKutf8} 				%Just in case I need asian languages.

%	Margins:
\usepackage[scale=0.75]{geometry}			   % Margins. Recommended 0.75.
\setlength{\hintscolumnwidth}{2.3cm}           % Dates column's width

%----------------------------------------------------------------------------------%
%            Personal Data 													[Prd]  %
%----------------------------------------------------------------------------------%

\firstname{Sebasti\'an}
\familyname{Torrente}

\title{Physics MS}

\address{43 Longridge House}{SE1 6QW Falmouth Road, London}

%	Contact:

%\mobile{629~787~950}						%Spanish mobile phone                            
\mobile{07423~632~073}
%\phone{(968)~463~074}						%Placeholder for future phone                            
%\fax{+34~(567)~890~123}						%Don't think I'll need it, but hey.

\email{sebastian.torrente@gmail.com}
%\homepage{infrafrequency.tumblr.com}
%\homepage{es.linkedin.com/in/sebastiantorrentecarrillo}
\extrainfo{Skype: sebastiantorrente}

%	Photo:	
%\photo[64pt][0.4pt]{foto3} 		%[photoheight][framewidth]{nameofthephoto.jpg}

%Quotes
%	\quote{"Life is too short to be little"}
%	\quote{"Nobody is safe from the power of Science!"}
%	\quote{"Breakdowns can create breakthroughs. Things fall apart so things can fall together"} 

%----------------------------------------------------------------------------------%
%	Content 																[Con]  %
%----------------------------------------------------------------------------------%
\begin{document}
%\begin{CJK*}{UTF8}{gbsn}                     %For CV in Chinese
\maketitle


%----------------------------------------------------------------------------------%
%	Profile																	[Pro]  %
%----------------------------------------------------------------------------------%

\section{Profile}
\cvitem{}{
I am a Physics graduate always in search of knowledge and challenges and with a strong interest in programming. With my degree finished, I've set my next goal in getting a first job or an internship where I can develop and hone a wide range of professional skills, complementing what I've learned during college and by myself. I offer an inquisitive and perseverant mind, happy to share his knowledge and lend a hand when needed.
}

%----------------------------------------------------------------------------------%
%	Education 																[Edu]  %
%----------------------------------------------------------------------------------%
\section{Education}

\cventry{2001-2013}
	{Licentiate Degree in Physics}
	{University of Murcia}
	{Murcia}
	{}
	{Equivalent to a BS+MS. Specialization in \textit{'Electronics and automatics.'}
	 With also computational physics and astrophysics classes due to personal interest.
	\begin{itemize}
		\item Student intern in the class 'General Relativity and Cosmology'.
%		\item Recovered from the big mistakes and personal problems I've had during 
%		my first years in college.
		\item Volunteered to edit and typeset the compilation book with all the 
		papers wrote for the Nuclear Physics class, which will be published. I got 
		more experience with \LaTeX, rewriting and turning the 
		papers into a single book with an unified style.
		\item Staff member in the \textit{\href{http://www.um.es/fispac/}
		{FISPAC}} stand during \textit{\href{http://www.f-
		seneca.org/secyt10/home.php}{SECyT 2010}}. Contributed with the creation of 
		posters and flyers and also giving divulgation speeches about astro and 
		particle physics to a wide variety of public. Solved the shortage of material 
		by creating a mailing list. Everyone interested was welcome to register so we 
		could send them digital versions of the posters. Thanks to this, FISPAC got 
		in contact with a lot of high schools for future divulgation activities.
	\end{itemize}	
	}

%----------------------------------------------------------------------------------%
%	Additional Education													[aEdu] %
%----------------------------------------------------------------------------------%

\section{Additional education}

% Sample Entry:
%\cventry{year--year}
	%{Degree}
	%{Institution}
	%{City}
	%{\textit{Grade}}
	%{Description}

\cventry{2012}{\href{https://6002x.mitx.mit.edu/}
	{6.002x Circuits and Electronics}}
	{MITx}
	{}
	{}
	{I wanted to brush up my knowledge in electronics and explore the option of online education.} 
	%Verification code: \newline{} \url{https://verify.edxonline.org/cert/e9af31ea9ce7435fa021ca908b4bfeac}}

\cventry{2012}
	{\href{https://www.edx.org/courses/BerkeleyX/CS169.1x/2013_Spring/about}
	{CS169.1x and 2x: Software as a Service}}
	{Edx}
	{}
	{}
	{It was a good way to learn Agile Development techniques and also give a try to Ruby and Rails. After this course I added Git into my arsenal of tools.} 
	%Verification codes: \newline{} \url{https://verify.edx.org/cert/61e969054b3e4edeacb84805c41d4301} and \newline{} \url{https://verify.edx.org/cert/29cf1609f4c540c38f0eea4184ff9670}}

\cventry{2012}
	{\href{https://6002x.mitx.mit.edu/}
	{CS188.1x: Artificial Intelligence}}
	{Edx}
	{}
	{}
	{Took this course out of curiosity and because it requires some Python programming I could use as training.} 
	%Verification codes: \newline{} \url{https://verify.edx.org/cert/2907f427165e4e3d85036db258752aab}}

%{Astroparticles: ¿De que esta hecho el universo?}
	%\cventry{2010}
	%{Universidad de Alcala}
	%{Sigüenza}{}
	%{Summer Course, 25 hours.}

%\cventry{2008}
%	{\href{http://biplot.usal.es/verano/analisis-estadistico-de-dat.html}
%	{Statistical Analysis of Data with SPSS}}
%	{University of Salamanca}
%	{Salamanca}
%	{}
%	{Summer Course, 30 hours.}

\cventry{2008}{\href{http://mat.uab.cat/~oalgebras/}
	{Schools of Mathematics Lluis Santal\'o. Aspect of operator algebras and applications}}
	{International University Men\'endez Pelayo}
	{Santander}
	{}
	{30 hours summer course. My goals were to meet people, learn some state of the art math and practice my english.}

%\cventry{2007}
	%{\href{http://cursosdeverano.unican.es/cursos-anteriores/Paginas/Detalle-curso.aspx?p_id=1094}
	%{Artificial Vision}}
	%{University of Cantabria}
	%{Santander}
	%{}
	%{Summer Course, 20 hours.}

\cventry{2007}
	{Quantum computation and topological orders}
	{Complutense University of Madrid}
	{San Lorenzo del Escorial}
	{}
	{Summer Course, 30 hours. Same reasons as above but with quantum physics.}

%\cventry{2006}
	%{Hidrogen and fuel batteries: energy base for the XXI century}
	%{UNED}
	%{El Barco de avila}
	%{}
	%{Summer Course, 18 hours.}

%\cventry{2006}					{\href{http://portal.uned.es/portal/page?_pageid=93,22547880&_dad=portal&_schema=PORTAL&idCurso=042}		{Process Control: Applications in MATLAB-SIMULINK}}
%	{UNED}
%	{\'Avila}
%	{}
%	{Summer Course, 35 hours.}

%\cventry{1992-2001}
%	{Bachelor of Music}
%	{Conservatory of Lorca}
%	{Lorca}
%	{}
%	{Violin}

%----------------------------------------------------------------------------------%
%	Master Thesis																   %
%----------------------------------------------------------------------------------%

%\section{Master Thesis}
	%\cvitem{titulo}{\emph{Title}}
	%\cvitem{sinodares}{Sinodares}
	%\cvitem{descripcion}{Brief Descrition}

%----------------------------------------------------------------------------------%
%	Experience																[Exp]  %
%----------------------------------------------------------------------------------%

\section{Experience}
%\subsection{Vocational}

\cventry{2006--Present}
	{Freelance Card Game Designer}
	{ \textit{\href{http://edgeent.com}{Edge Entertainment}}}
	{}
	{}
	{Co-creator and Main Designer of the card game 'Crisis: Tokyo', of future release in 
	three languages.
\begin{itemize}
	\item Won the board game design contest (card game category) thanks in no small 
	part to the creation of a convincing prototype and a good pitch.
	\item Learned all the aspect of the design and management of a product, given 
	that \textit{Edge Entertainment} gave us a lot of control (and responsibility) 
	over all the process and the final say on every aspect.
	\item It was also a chance to take a lot of different roles and to collaborate 
	with external teams: editors, illustrators, translators, beta testers...
\end{itemize}}

%\cventry{2011--2012}
%	{Editing and typesetting}
%	{\href{http://www.um.es/fispac/}{FISPAC}}
%	{Murcia}
%	{}
%	{Edition, typesetting and correction in \LaTeX{} of the book \textit{'Progress on particle and nuclear physics'} (provisional title)\newline{}}

%\cventry{September 2010}
%	{Collaboration in the FISPAC stand in the Murcia's Science Fair 2010}
%	{FISPAC}
%	{Murcia}
%	{}
%	{Design of posters and other material for the stand.}

%----------------------------------------------------------------------------------%
%	Languages																[Lan]  %
%----------------------------------------------------------------------------------%

%	Example liine:
%\cvitemwithcomment{Idioma 3}{nivel}{Comentario}

\section{Languages}
\cvitemwithcomment{Spanish}
	{Native}
	{Great level with a rich vocabulary, even for a native.}

\cvitemwithcomment{English}
	{Fluent spoken and written}
	{\href{https://dl.dropbox.com/u/87894135/StatementOfResult.pdf}
	{ESOL Cambridge CAE Grade A}}

\cvitemwithcomment{French}
	{Basic}
	{Learned during high school, I plan to pick it up again.}

%----------------------------------------------------------------------------------%
%	Computer Skills															[Com]  %
%----------------------------------------------------------------------------------%

\section{Computer Skills}

\cvitemwithcomment{Python}
	{2.7, 3.3, IPython}
	{Preferred language, currently learning Numpy and Scipy}

\cvitemwithcomment{\LaTeX}
	{MikTex and Texmaker}
	{Main choice for typesetting}

\cvitemwithcomment{Fortran}
	{FORTRAN77}
	{Used in computational physics, switching to Numpy}

%\cvitemwithcomment{MATLAB}
%	{Octave}
%	{}

\cvitemwithcomment{Other}
	{MATLAB, C, R, Git(Github)}
	{\href{https://github.com/SebastianTorrente/}
	{https://github.com/SebastianTorrente/}}

%\cvitemwithcomment{Learning}{Ruby, Rails, Django}{}

%\cvitemwithcomment{Tools}
%	{Git (\href{https://github.com/SebastianTorrente/}
%	{Github})}
%	{\href{https://github.com/SebastianTorrente/}
%	{https://github.com/SebastianTorrente/}}

%----------------------------------------------------------------------------------%
%	Interests																[Int]  %
%----------------------------------------------------------------------------------%

\section{Interests}

\cvitem{Electronics}
	{Design, assembly and soldering of small electronics devices. Last project: an arcade stick with detachable cable and compatible with a wide range of systems.}

\cvitem{Learning}
	{Acquiring new skills. Currently learning Ruby, and TkInter.
	Once I settle down I plan to learn German, French and some martial art as a 
	physical activity.}

%\cvitem{Game design}
%	{Mainly card and boardgames. As stated above, I won a contest in card game design.}

\cvitem{Hobbies}
	{Taking online and summer courses, travel, cooking, organising science fiction and martial arts cinema sessions, read, trade and discuss comics and novels and make small fighting game tournaments with friends.}

%\cvitem{hobby 3}{Descripcion}

%----------------------------------------------------------------------------------%
%	Personal Skills															[Per]  %
%----------------------------------------------------------------------------------%

\section{Personal Skills}

\cvitem{Lateral Thinker}
{Crazy ideas may not be the correct ones all the time, but they are always interesting and worth exploring.}

\cvitem{Problem solver}
{With a "let's fix it" attitude and very enthusiastic about it. Doubly so if I have to learn new skills to tackle the problem. I would like to insist about it: I really love to learn new skills, even if they do not seem practical at first.}

\cvitem{Team worker}
{Silver tongue with great exposition skills. Supportive to others, happy to share and support ideas and fun to have around.}

\cvitem{Curious and polymorphic}
{I just cannot resist the temptation of trying my hand at tasks I've never done before and figuring out how things work.}

%\cvitem{Perseverant}
%{Even in the direst circumstances I still push on.}

%\cvitem{Other}
%{Driving license.}

%-------------------------------------------------------------------------------------

\renewcommand{\listitemsymbol}{-~}            %Change the symbol in lists

%\section{Extra 2}
%\cvlistdoubleitem{Tema 1}{Tema 4}
%\cvlistdoubleitem{Tema 2}{Tema 5\cite{book1}}
%\cvlistdoubleitem{Tema 3}{}

% Las publicaciones tomadas de un archivo de BibTeX sin usar multibib\renewcommand*{\bibliographyitemlabel}{\@biblabel{\arabic{enumiv}}}

%\nocite{*}
%\bibliographystyle{plain}
%\bibliography{publications}                   % 'publications' es el nombre del archivo BibTeX

% Las publicaciones tomadas de un archivo BibTeX usando el paquete multibib
%\section{Publicaciones}
%\nocitebook{book1,book2}
%\bibliographystylebook{plain}
%\bibliographybook{publications}              % 'publications' es el nombre del archivo BibTeX
%\nocitemisc{misc1,misc2,misc3}
%\bibliographystylemisc{plain}
%\bibliographymisc{publications}              % 'publications' es el nombre del archivo BibTeX

%\clearpage\end{CJK*}                          % si esta redactando su CV en chino usando CJK, \clearpage es requerido por fancyhdr para que funcione correctamente con CJK, aunque esto eliminara la numeracion de pagina al dejar \lastpage como no definido
\end{document}


%%%%%%%%%%%%%%%%%%%%%%%%%%%%%%%%%%%%%%%%%%%%%%%%%%%%%%%%%%%%%%%%%%%%%%%%%%%%%%%%%%%%%%